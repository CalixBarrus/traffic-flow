\documentclass[12pt]{article}

\usepackage{graphicx}
\usepackage{amssymb,amsmath,amsthm}




% the following commands set the size of printed page
\setlength{\textheight}{680pt} \setlength{\topmargin}{-50pt}
\setlength{\textwidth}{500pt}
\setlength{\evensidemargin}{-10pt}
\setlength{\oddsidemargin}{-10pt}
%%%%%%%%%%%%%%%%%%%%%%%%%%%%

\begin{document}

\title{Place title of Group Project Here}
\author{Riley Davidson, Laren Edwards, Dallan Olsen, Calix Barrus}

%% comment out next command to put today's date after names of group members, or put a desired day in the parethesis
\date{}

\maketitle

\begin{abstract}
Place Abstract Here.
\end{abstract}

%% First Section
\section{Background/Movitation}
Put your background motivation stuff here. You might reference something material here like \cite{ref1}.

%% Second Section 
\section{Modeling}

Beginning with the equation for traffic flow $u_t + V_\infty \left(1 - \frac{2u}{u_{\infty}} \right) u_x = 0$, we will try to find an explicit solution using the method of characteristics. 

Suppose the solution $u$ is some curve parameterized by $s$ as $u(x(s), t(s))$. Taking the derivative of $u$ with respect to $s$ gives 
\begin{align*} % in the book, the start off with the claim that this equation is = 0 or = f. How do we justify setting u' equal to the differential equation, and also equal to the RHS of the PDE? 
    u'(x(s), t(s)) = \frac{\partial u}{\partial x} \frac{\partial x}{\partial s}  + \frac{\partial u}{\partial t} \frac{\partial t}{\partial s} = u_x \frac{\partial x}{\partial s}  + u_t \frac{\partial t}{\partial s}
\end{align*}
and matching terms with the original differential equation gives us that 
\begin{align*}
    \frac{\partial x}{\partial s}  = V_\infty \left( 1 - \frac{2}{u_\infty} u \right) \\
    \frac{\partial t}{\partial s} = 1 \\
    \frac{\partial u}{\partial s} = 0 .
\end{align*}

Using the fact that $\frac{\partial t}{\partial s} = 1$, we have that 
\begin{align*}
    \frac{\partial x }{\partial t} = V_\infty (1 - \frac{2}{u_\infty} ) \\
    x(t) - x(0) = \int_{0}^{t} V_\infty(1 - \frac{2}{u_\infty} u) dt \\
    x = V_\infty(1 - \frac{2}{u_\infty} u) t + x(0) \\
    t = \frac{1}{  V_\infty(1 - \frac{2}{u_\infty} u) }x - x(0)
\end{align*}
where to evaluate the integral, we used the fact that $\frac{\partial u}{\partial s}$ is $0$, and so the characteristics are constant along the plane. This yields a few cases for the characterstic curves depending on what value of $u$ the curve takes. Based on the physical properties of our model, we have that $u \in [0, u_\infty]$. The slope of the characteristics changed depending on the value of $u$, specifically the slope $m = > 0$ when $u \in [0, \frac{u_\infty}{2}) $, the slope is vertical or undefined when $u = \frac{u_\infty}{2}$, and $m < 0$ when $u \in ( \frac{u_\infty}{2}, u_\infty]$.

%% Third Section
\section{Results}

%%Fourth Section
\section{Analysis/Conclusions}

\begin{thebibliography}{99}
\bibitem{ref1} First reference goes here
\end{thebibliography}

\end{document}
